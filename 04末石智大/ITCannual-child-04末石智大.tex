\subsection{研究報告(末石 智大)}

 高速画像処理および高速光学系制御を用いた、動的検査技術とヒューマンインターフェースに関する研究を昨年度に引き続き実施した。

動的検査技術は、実世界に存在する複雑な現象を適切にデータ化・活用する技術であり、本年度は特にマイクロサッカードと呼ばれる眼球微振動などを対象として実施した。静止状態における検査技術は数多くあるが、時間効率が低い・被験者を拘束する負荷がかかるなど、動的状態への検査技術の発展の期待は大きいと考えられる。あご台を用いずリラックスした状態の人間のマイクロサッカード検出に向けて、回転ミラーや液体可変焦点レンズなどの光学素子を高速に制御し、運動対象の高解像度撮影を達成することで運動物体への検査のための基礎技術開発・発展に努めている。昨年度までに開発していたマイクロサッカード計測システムの改良に加え、微振動を抽出するための解析アルゴリズム開発や、定量的評価のための動的眼球模型の提案・開発などの成果を実現した。特に動的眼球模型は眼球特性・マイクロサッカード動作を適切に模倣した光学・制御設計により、画像計測システムとしてのデータベース構築にも役立つ。

ヒューマンインターフェースに関しては、新たに高速手指トラッキング用の指輪マーカーを開発し、国際学会の口頭発表ならびに国内学会での受賞を達成した。一般に人体動作をデータ化するモーションキャプチャは体表にドットマーカーを複数装着することで計測を実現するが、このマーカーの装着持続性が低いことに着目し、複数楕円から成る指輪型マーカーを提案・開発し、高速トラッキング及びダイナミックプロジェクションマッピングへの適用性を示した。また、昨年度に引き続き球体へのダイナミックプロジェクションマッピングや高速アイトラッキングの改良も実施し、非対称マーカー埋め込みや注視点指向高速映像投影との連携などを新たに加えた内容で、それぞれ論文誌に掲載された。

本年度は総じて、眼球や手指など高速な身体挙動に着目した高速センシングおよび高速情報呈示に関する新たな計測制御技術を開発し、かつデータ解析や実世界応用へと繋がる成果を創り出した。