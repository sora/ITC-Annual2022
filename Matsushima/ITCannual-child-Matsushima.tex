\subsection{研究報告(松島 慎)}

本節では2021年度の松島研究室の研究活動について報告する。
弊研究室では知識発見のための機械学習手法の効率的な計算手法についての研究を推進してきた。
与えられたデータを用いて入力から出力を予測する方法論としての機械学習は特に深層学習モデルの学習手法の発展を通じて
高い精度での予測を実現できることが様々な分野で報告されている。
これらは画像の識別や多言語間の翻訳など、公開されたデータが豊富にあるタスクにおいて複雑な関数を学習する方法論である。
一方で、複雑な関数を機械が学習できることは必ずしも我々の画像や言語に対する理解を深めるわけではない。
現代社会には様々なデータが蓄積されており、データを上述の意味で利用するだけでなく、隠れた法則性や生成原理などの理解に結び付く属性間の関係を抽出することが求められる。
そのため、
我々はデータの関係を理解するのに有用なモデルとして以下の3つの分野に関する研究を行っている。
\begin{itemize}
    \item 一般化加法モデルに関する研究
    \item 組合せ線形モデルに関する研究
    \item 部分空間クラスタリングに関する研究
\end{itemize}
2021年度はさらに統計的因果探索と呼ばれる手法群について調査・研究を開始した。

統計的因果探索とは、変数間の単純な相関関係だけでなく因果関係にも着目し、属性間の構造を推定する手法である。
統計的因果探索において因果関係とは介入と呼ばれる変数の操作が他の変数の分布へどう影響を及ぼすかを表す概念である。例えば、特定の薬を服用したかどうかの変数に着目した時、
無作為に選んだ人に薬を服用させた場合、この変数への介入操作が行われたとみなすことができる。
その集団に関する他の変数の分布は介入後分布と呼ばれ条件付き分布は一般に別物である。

我々は統計的因果探索の手法の中でも、介入操作が行われたデータを使わずに、変数群の同時分布から無作為に抽出されたとみなされるデータから因果関係を推定する手法の調査・研究を行った。
前述した介入の定義によると介入操作の結果を得ない限りは因果関係の推定は不可能のように考えられるが、
データの生成過程が構造方程式モデルで記述できると仮定した場合は、同時分布からのデータのみから因果関係の推定が可能である場合がある。

一般に確率変数をノードとする有向グラフ$G=(V,E)$に関する構造方程式モデルとは、各$e_j$を互いに独立な確率変数とし、
各確率変数$X_{j}$を以下のような方程式系で表すものである:
\begin{align*}
    X_{j}=f_{j}\left(\mathbf{PA}(X_{j}), e_{j}\right), \quad j=1, \ldots, d
\end{align*}
ここで、$\mathbf{PA}(X_j)$は$G$で$X_j$の親となる確率変数の集合、$f_j$は考えている確率空間において可測な任意の関数である。

DAG (Directed Acyclic Graph)である$G=(V,E)$に関する構造方程式モデルが識別可能であるとは、同時分布からのデータのみから因果関係の推定、すなわちグラフ$G$の推定が可能であることを意味し、
特にLiNGAMモデルは識別可能な構造方程式モデルの部分集合としてよく知られているものである\cite{S}。本年度はLinGAMモデルの学習手法を検討し、実験的に既存手法を上回る精度を持つ手法の設計・開発を行った。今後この手法のさらなる精度改善や学習効率化を検討し、対外発表を行う予定である。

\begin{thebibliography}{9}
\bibitem{S} 
Shimizu, Shohei, et al. "A linear non-Gaussian acyclic model for causal discovery." Journal of Machine Learning Research 7.10 (2006).
%\bibitem{F} 
\end{thebibliography}
