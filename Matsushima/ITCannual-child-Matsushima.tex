\subsection{研究報告(松島 慎)}

本節では2021年度の松島研究室の研究活動について報告する。
弊研究室で解釈可能な機械学習手法の効率的な計算手法についての研究を推進してきた。
2021年度は特に統計的因果探索と呼ばれる手法群について調査・研究を開始した。


大量のデータから複雑な関数を推定することにより画像データや言語データを予測したり生成したりするなど、表層的に駆使することは可能になってきた。
しかしながら、複雑な関数を機械が学習できることは必ずしも我々の画像や言語に対する理解を深めるわけではない。現代社会には様々なデータがあり、データを上記の意味で駆使するだけでなく、データの隠れた法則性や生成原理などの理解に結び付く属性間の関係を抽出することが求められる。

機械学習は予測や分類など表層的なデータの駆使の方法論であるだけでなく、
データの関係を明らかにして人間の理解を助けるための方法論でもあり、
特にデータ保持者がデータを理解するのに有用なモデルを研究する重要性は絶対的にも相対的にも増してきている。
我々はデータの関係を理解するのに有用なモデルとして以下の3つの分野に関する研究を行った。
\begin{itemize}
    \item 一般化加法モデルに関する研究
    \item 組合せ線形モデルに関する研究
    \item 部分空間クラスタリングに関する研究
\end{itemize}
