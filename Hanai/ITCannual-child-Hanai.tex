\subsection{研究報告(華井 雅俊)}

 本節では、2021年度の華井雅俊の研究活動について報告する。2021年9月の本学着任から、グラフニューラルネットワークとその物性予測問題への応用に関する研究に取り組んでいる。

電池、半導体、触媒、医薬品などの材料開発・材料研究の全般において、膨大にある候補材料のさまざまな物性を比較解析することが不可欠であるが、それら候補全てを実際に作り検証することは現実的でない。そのため分子構造などの比較的簡単に得られる物質情報から目的の物性を予測・計算することが重要である。近年では、分子構造(グラフ)データとグラフニューラルネットワークを利用した物性値予測モデルの研究が盛んになってきている。特に2021年度はStanford Universityが取りまとめるOpen Graph Benchmark (OGB) やCMUとFacebookが主導するOpen Catalyst Project (OCP) などの機械学習系研究コミュニティのコンペティションで物性予測問題が取り上げられた初めての年であった。

一般に、ある物性値が広範囲な材料群に対し既知である場合予測モデルを構築することが可能となるが、しかし一方で、多くの物性値においては既知である材料が少数であり学習データが不足しているため、実用精度の予測モデルを構築することは難しい。同一の物性であってもパラメータや実験条件が共通化されていないと予測モデルの構築は難しいことが知られ、既存の物性予測の研究では、共通の条件で整理された大規模データが主に利用される(例えば、上のコンペティションなど)。小規模に限定されるデータ、例えば計算コストの膨大なシミュレーション値や実験データ、において、機械学習の利用は限定的であり、大きな研究課題の1つとなっている。

我々の研究チームはこのような少規模データに着目し研究を開始した。2021年度下半期は新手法提案への準備としてデータの収集に注力し研究を行った。機械学習分野や材料研究分野で用いられるオープンデータに加え、同学の工学部の研究チームへコンタクトし、スパコンスケールの計算資源を利用し得られた高価なシミュレーション値や実際の実験データに関してヒアリングを行い、データ収集を開始した。
また、本部門で開発のすすめるmdxにおいては材料系研究への利用促進を行っており、本研究の中間報告として第20回ナノテクノロジー総合シンポジウムにて発表し、IEEE IC2E 2022への投稿論文にて材料系研究におけるクラウド基盤の利活用をまとめた。

% 2021年度は主に、分野の調査と



