\subsection{研究報告(石川 正俊)}

 並列処理を内蔵した高速ビジョンの基本アーキテクチャの開発を処理構造の観点のみならず、 応用システムを考慮した設計を行っている。
知能ロボット等の機械システムの視覚フィードバック制御に関しては、 現状のプレイバック制御の構造に変えて、 ローカルに高速視覚フィードバックを導入したダイナミック補償を導入した構造を提案し、 いくつかのシステムを構築した。その1つの実装形態としてとして、 人間機械協調システムの提案を行っている。

また、 高速ビジョンから得られるデータに対応するカラー高速プロジェクターの高速プロジェクター並びにその応用システムの開発を行った。高速プロジェクターを用いることにより、 高速に変化する対象や高速に移動する対象に対して投影が可能となる。これをダイナミックプロジェクションマッピングと呼び、 高速ビジョンによる高速三次元形状計測を入力と高速光軸制御や高速可変焦点レンズと組み合わせることにより、 高速対象を媒体とした新たなヒューマンインターフェイスが実現可能となる。

視覚フィードバックを伴うヒューマンインターフェイスにおいて、 実世界からディスプレイ上の表示までの遅延が人間のタスクに対する影響の評価に対して、 高速ビジョンと高速プロジェクターを用いることにより、 高い時間分解能で遅延を設定できるシステムを提案・実装した。これにより、 具体的なタスクに対する遅延の影響の計測・評価が可能となる。

これらの研究も含め、 高速画像処理を用いた高速知能システムを実現するための理論、 デバイス、 システムアーキテクチャの研究とともに、 応用システムの開発を行っている。
