\subsection{研究報告(姜 仁河)}
本節では2021年度の姜仁河の研究活動について報告する。近年、都市のスマート管理、スマートシティは新しい科学技術分野として各国の学術界、産業界および各国政府から非常に重視されている。モバイルデータ、IoTセンサデータ、衛星画像、交通プローブデータ、災害データなどダイナミックなリアルタイム時空間ビッグデータが入手可能な環境が急速に整いつつあり、健康や医療サービスデータ、購買履歴データ、物流・商流などの経済データも積極的に活用されている。これらのデータを統合した形で人々や企業の活動、交通・物流・商流から都市の拡大・環境変化、社会経済システムの変質・変動までを包含するデジタル社会空間のあらゆる課題を解決する。これを目的にして、引き続き2021年度、私は時空間データインテリジェンスについて研究活動を行ってきた。

時空間データを一定の時間間隔・空間単位で集計すると、汎用的に3Dテンソル「T×N×C」(Tはタイムステップの数、Nは交差点やリンクやメッシュグリッドや不規則な区域の数、Cは情報量の数)で表現できる。時間軸Tから見ると、将来の予測値は最近の観測値と過去に現れた定期的なパターンに依存する。空間軸Nから見ると、ある交差点や区域内の交通量は近隣や遠くにあるものに影響される。このような3Dテンソル「T×N×C」に含まれる超複雑な時空間依存関係をモデリングするために、一連の深層学習モデル(Temporal Convolution Network、Recurrent Neural Network 、Graph Convolution Network、Attention Mechanism)を設計・開発し、群衆密度、タクシー・シェアサイクル需要、救急車需要、新型コロナ感染者数を高精度で予測可能にした。関連成果はデータサイエンス分野のトップカンファレンスKDD2021、ICDE 2021、ECMLPKDD2021、CIKM2021、AAAI2022\cite{JIANG2-1,JIANG2-2,JIANG2-3,JIANG2-4,JIANG2-5,JIANG2-6}及びトップジャーナルACM TIST 2022、ACM TKDD 2021、IEEE TKDE2021、IEEE TVCG2021\cite{JIANG1-1,JIANG1-2,JIANG1-3,JIANG1-4,JIANG1-5}にて発表された。特に、交通流モデリング・予測の深層学習ベンチマークDL-Traff\cite{JIANG2-3}は、国際トップカンファレンスCIKM 2021において最優秀リソース論文賞(Best Resource Paper Runner-Up Award)を受賞した。また、Yahoo Japan Researchとの共同研究によって、オープンデータサイエンスの促進に努めた。個人情報・プライバシー問題に配慮したうえで、メッシュベースの東京都・大阪市の群衆密度・移動データをデータサイエンス分野トップジャーナルであるIEEE TKDE2021\cite{JIANG1-4}において公開した。

