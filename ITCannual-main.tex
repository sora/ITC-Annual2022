\documentclass[11pt]{jarticle}
\usepackage{ITCannual}
\usepackage{amsmath}
\usepackage{amssymb}
\usepackage{times}
\usepackage{graphicx}

%\usepackage[style=numeric]{biblatex}

\title{データ科学研究部門 研究報告}
\author{小林博樹, 鈴村豊太郎, 松島慎, 空閑洋平, 姜仁河, 川瀬純也, 華井雅俊, XXX, 他X名}

\begin{document}
\maketitle

\section{データ科学研究部門 概要}

%データ科学研究部門では、2019年度、助教 1 名。2020年度、教授2名(特任教授1名)、准教授2名(特任准教授1名)、講師3名(特任講師3名)、助教 6 名(助教1名は途中から兼任として在籍。助教1名は11月着任。特任助教3名)が在籍した。同部門のメンバーは専任教員と特任教員の2つのグループから成る。専任教員はそれぞれが独立して研究活動を行うグループで、特任教員は石川特任教授を中心とする石川研究室グループである。 


\subsection{専任教員グループの研究テーマ}

%計算機を介した人と生態系のインタラクションの研究(小林)\\
%解釈可能な機械学習手法の効率的な計算手法についての研究(松島)\\
%データ駆動型人文学研究の実践(中村)\\
%データ駆動型知能に基づくアーバンコンピューティング(姜)\\
%野生動物ワイヤレスセンサネットワーク実証実験基盤構築に向けた研究(川瀬)\\


\subsection{石川研究室全体の研究活動概要}

\section{データ科学研究部門 成果要覧}
\begin{招待講演}{1}

\bibitem{04末石智大01}
末石智大:高速光学系制御に基づくダイナミックビジョンシステムとその応用,第3回産業ロボット関連技術の標準化学術研究会,2022.

\bibitem{ykuga36619767}
空閑洋平, mdx: データ活用社会創成プラットフォーム構築の現状と今後, CloudWeek 2021@Hokkaido University, 2 Sep, 2021.

\bibitem{ykuga36619729}
空閑洋平, NetTLP: ハードウェアと協調動作可能なソフトウェアPCIeデバイス開発環境, 情報処理学会 FIT 情報科学技術フォーラム トップコンファレンスセッション, 26 Aug, 2021.



%\bibitem{suzumura-mdx2022}
%Toyotaro Suzumura, Akiyoshi Sugiki, Hiroyuki Takizawa, Akira Imakura, Hiroshi Nakamura, Kenjiro Taura, Tomohiro Kudoh, Toshihiro Hanawa, Yuji Sekiya, Hiroki Kobayashi, Shin Matsushima, Yohei Kuga, Ryo Nakamura, Renhe Jiang, Junya Kawase, Masatoshi Hanai, Hiroshi Miyazaki, Tsutomu Ishizaki, Daisuke Shimotoku, Daisuke Miyamoto, Kento Aida, Atsuko Takefusa, Takashi Kurimoto, Koji Sasayama, Naoya Kitagawa, Ikki Fujiwara, Yusuke Tanimura, Takayuki Aoki, Toshio Endo, Satoshi Ohshima, Keiichiro Fukazawa, Susumu Date, Toshihiro Uchibayashi, "mdx: A Cloud Platform for Supporting Data Science and Cross-Disciplinary Research Collaborations", https://arxiv.org/abs/2203.14188

\bibitem{suzumura-nci2021}
鈴村豊太郎, “mdx: A platform for the data-driven future”、オーストラリア国立研究所NCI(National Computational Infrastructure)-Fujitsu HPC, Cloud and Data Futures Workshop, 2022年02月

\bibitem{suzumura-nanotec2021}
鈴村豊太郎, "データ活用社会創成プラットフォームmdxにおけるマテリアルズ・インフォマティクス研究・共創に向けて", 第20回ナノテクノロジー総合シンポジウム, 2022年01月

\bibitem{suzumura-canon2021}
鈴村豊太郎, "人工知能を支えるグラフニューラルネットワークの最新動向", 2021年度キャノングローバル戦略研究所主催「経済・社会との分野横断的研究会」, 2021年12月


\end{招待講演}

\begin{招待論文}{1}

\bibitem{02早川智彦01}
早川智彦,石川正俊,亀岡弘之:時速100km走行でのトンネル覆工コンクリート高解像度変状検出手法,建設機械施工,vol.73,no.8,pp.19-23,2021.

\bibitem{02早川智彦02}
早川智彦,望戸雄史,石川正俊,大西偉允,亀岡弘之:【大臣賞】時速100km走行での覆工コンクリート高解像度変状検出手法,土木施工,vol.62,no.7,p.146,2021.

\bibitem{04末石智大02}
Yuri Mikawa,Tomohiro Sueishi,Yoshihiro Watanabe,and Masatoshi Ishikawa:Dynamic Projection Mapping for Robust Sphere Posture Tracking Using Uniform / Biased Circumferential Markers,2022 IEEE Conference on Virtual Reality and 3D User Interfaces (VR2022),(TVCG Invited)2022.


\end{招待論文}

\begin{受賞}{1}

\bibitem{03黄守仁01}
村上健一,黄守仁,石川正俊,山川雄司:高速ビジュアルフィードバックを用いた高速3次元位置補償システムの開発,第22回計測自動制御学会システムインテグレーション部門講演会(SI2021),講演会論文集,pp.1403-1405,優秀講演賞,2021.

\bibitem{04末石智大03}
末石智大,石川正俊:手指高速トラッキングに向けた楕円群指輪マーカーの開発,第22回計測自動制御学会システムインテグレーション部門講演会(SI2021),講演会論文集,pp.1382-1387,優秀講演賞,2021.

\bibitem{kobayashi1-1}
小林博樹:情報通信技術の導入が困難なインフラ圏外空間を対象とした情報デザインとIoTの研究,ドコモ・モバイル・サイエンス賞 社会科学部門 優秀賞,2021/9.

\bibitem{kobayashi1-2}
Hill Hiroki Kobayashi, Radioactive Live Soundscape, Winner, Universal Design Expert, Institute for Universal Design KG, Germany, 2021/05.
\bibitem{JIANG3-1}
Renhe Jiang et al., "DL-Traff: Survey and Benchmark of Deep Learning Models for Urban Traffic Prediction", 30th ACM International Conference on Information and Knowledge Management (CIKM), Best Resource Paper Runner Up, 2021.

\end{受賞}

\begin{著書}{1}

\bibitem{05宮下令央01}
宮下令央:有名論文ナナメ読み「Shader Lamps: Animating Real Objects With Image-Based Illumination」,情報処理学会,情報処理,Vol.62,No.5 ,2021.


\end{著書}

\begin{雑誌論文}{1}

\bibitem{01石川正俊01}
川原大宙,妹尾拓,石井抱,平野正浩,岸則政,石川正俊:輪郭情報に基づくテンプレートマッチングを用いた重畳車両の高速トラッキング,計測自動制御学会論文集,58巻,1号,pp.21-30,2022.

\bibitem{01石川正俊02}
小山佳祐,堀邊隆介,安田博,万偉偉,原田研介,石川正俊:ワンボード・USB給電タイプの高速・高精度近接覚センサの開発とプリグラスプ制御の解析,日本ロボット学会誌,Vol.39,No.9,pp.862-865,2021.

\bibitem{01石川正俊03}
Masahiro Hirano,YujiYamakawa,Taku Senoo and Masatoshi Ishikawa:An acceleration method for correlation-based high-speed object tracking,Measurement Sensors,Vol.18,Article No.100258,2021.

\bibitem{01石川正俊04}
Masahiko Yasui,Yoshihiro Watanabe,and Masatoshi Ishikawa:Wide viewing angle with a downsized system in projection-type integral photography by using curved mirrors,Optics Express,Vol.29,Issue8,pp.12066-12080,2021.

\bibitem{01石川正俊05}
Ruimin Cao,Jian Fu,Hui Yang,Lihui Wang,and Masatoshi Ishikawa:Robust optical axis control of monocular active gazing based on pan-tilt mirrors for high dynamic targets,Optics Express,Vol.29,No.24,pp.40214-40230,2021.

\bibitem{02早川智彦03}
Kenichi Murakami,Tomohiko Hayakawa,and Masatoshi Ishikawa: Hybrid surface measuring system for motion-blur compensation and focus adjustment using a deformable mirror,Applied Optics,vol.61,Issue2,pp.429-438,2022.

\bibitem{02早川智彦04}
Yuki Kubota,Yushan Ke,Tomohiko Hayakawa,Yushi Moko,and Masatoshi Ishikawa:Optimal Material Search for Infrared Markers under Non-Heating and Heating Conditions,Sensors,Vol.21,Issue 19,Article No.6527,pp.1-17,2021.


\bibitem{04末石智大04}
Yuri Mikawa,Tomohiro Sueishi,Yoshihiro Watanabe,and Masatoshi Ishikawa:Dynamic Projection Mapping for Robust Sphere Posture Tracking Using Uniform/Biased Circumferential Markers,IEEE Transaction on Visualization and Computer Graphics,1-1,2021 (Early Access).

\bibitem{04末石智大05}
松本明弓,新田暢,末石智大,石川正俊:高速注視点推定を用いた広域高解像度投影システムの実現,計測自動制御学会論文集,Vol.58,No.1,pp.42-51,2022.

\bibitem{05宮下令央02}
井倉 幹大,宮下令央,山下 淳,石川 正俊,淺間 一:高速点滅LEDマーカと複数のRGB-Dセンサを用いた遮蔽領域を提示可能な任意視点重畳映像生成システム,精密工学会誌,Vol. 88,No.3,2022.

\bibitem{05宮下令央03}
Leo Miyashita,Akihiro Nakamura,Takuto Odagawa,and Masatoshi Ishikawa: BIFNOM Binary-Coded Features on Normal Maps,Sensors,Vol.21,No.10,Article No.3469,2021.


\bibitem{06田畑智志01}
Hongjin Xu,Lihui Wang,Satoshi Tabata,Yoshihiro Watanabe,and Masatoshi Ishikawa:Extended depth-of-field projection method using a high-speed projector with a synchronized oscillating variable-focus lens,Applied Optics,Vol.60,Issue 13,pp.3917-3924,2021.

\bibitem{08金賢梧01}
Hyuno Kim,and Masatoshi Ishikawa:Sub-Frame Evaluation of Frame Synchronization for Camera Network Using Linearly Oscillating Light Spot,Sensors,Vol.21,Issue 18,Article No.6148,pp.1-14,2021.

\bibitem{ykuga36595746}
Yukito Ueno, Ryo Nakamura, Yohei Kuga, Hiroshi Esaki, A NIC-driven Architecture for High-speed IP Packet Forwarding on General-purpose Servers, Journal of Information Processing, 30, pp226-237, 2022.

\bibitem{JIANG1-1}
Renhe Jiang, Zekun Cai, Zhaonan Wang, Chuang Yang, Zipei Fan, Quanjun Chen, Xuan Song, and Ryosuke Shibasaki, "Predicting Citywide Crowd Dynamics at Big Events: A Deep Learning System", ACM Trans. Intell. Syst. Technol. (TIST), 13, 2, Article 21, April 2022.
\bibitem{JIANG1-2}
Zipei Fan, Chuang Yang, Zhiwen Zhang, Xuan Song, Yinghao Liu, Renhe Jiang, Quanjun Chen, and Ryosuke Shibasaki, "Human Mobility-based Individual-level Epidemic Simulation Platform", ACM Trans. Spatial Algorithms Syst. (TSAS), 8, 3, Article 19, September 2022.
\bibitem{JIANG1-3}
Chuang Yang, Zhiwen Zhang, Zipei Fan, Renhe Jiang, Quanjun Chen, Xuan Song, Ryosuke Shibasaki, "EpiMob: Interactive Visual Analytics of Citywide Human Mobility Restrictions for Epidemic Control", IEEE Transactions on Visualization and Computer Graphics (TVCG), 2022.
\bibitem{JIANG1-4}
Renhe Jiang, Zekun Cai, Zhaonan Wang, Chuang Yang, Zipei Fan, Quanjun Chen, Kota Tsubouchi, Xuan Song, Ryosuke Shibasaki, "DeepCrowd: A Deep Model for Large-Scale Citywide Crowd Density and Flow Prediction", IEEE Transactions on Knowledge and Data Engineering (TKDE), 2021.
\bibitem{JIANG1-5}
Jinliang Deng, Xiusi Chen, Zipei Fan, Renhe Jiang, Xuan Song, and Ivor W. Tsang, "The Pulse of Urban Transport: Exploring the Co-evolving Pattern for Spatio-temporal Forecasting", ACM Trans. Knowl. Discov. Data (TKDD), 15, 6, Article 103, May 2021.
\bibitem{HNakamura1}
山下智也,宮本大輔,関谷勇司,中村宏, ``通信挙動に基づいたスキャン攻撃検
知'', 情報処理学会論文誌~デジタル社会の情報セキュリティとトラスト~特集
号,11pages, Vol.62,No.12,pp.67-83


\end{雑誌論文}

\begin{査読付}{1}


\bibitem{01石川正俊06}
Hiromichi Kawahara,Taku Senoo,Idaku Ishii,Masahiro Hirano,Norimasa Kishi and Masatoshi Ishikawa:High-speed tracking for overlapped vehicles using Instance Segmentation and contour deformation,2022 IEEE/SICE International Symposium on System Integration (SII2022),Proceedings,pp.730-735,2022.


\bibitem{01石川正俊07}
石川正俊:高速ビジョンを用いた高速知能ロボット,ロボット,No.263,pp.56-58 ,2021.

\bibitem{01石川正俊08}
石川正俊:情報科学技術の構造と情報教育,IDE 現代の高等教育,2021年8-9月号,No.633,pp.9-13,2021.

\bibitem{01石川正俊09}
Masahiro Hirano,Yuji Yamakawa,Taku Senoo,Norimasa Kishi,Masatoshi Ishikawa:Multiple Scale Aggregation with Patch Multiplexing for High-speed Inter-vehicle Distance Estimation,IEEE Intelligent Vehicles Symposium (IV),Proceedings,pp.1436-1443,2021.

\bibitem{01石川正俊10}
Hirofumi Sumi,Hironari Takehara,Jun Ohta,and Masatoshi Ishikawa:Advanced Multi-NIR Spectral Image Sensor with Optimized Vision Sensing System and Its Impact on Innovative Applications,2021 Symposium on VLSI Technology ,2021 Symposium on VLSI Technology Digest of Technical Papers,JFS4-8,pp.1-2,2021.

\bibitem{01石川正俊11}
Masahiko Yasui,Yoshihiro Watanabe,and Masatoshi Ishikawa:Dynamic and Occlusion-Robust Light Field Illumination,ACM SIGGRAPH ASIA 2021 Posters (SIGGRAPH ASIA 2021),Proceedings,Article No.35,pp.1–2,2021.

\bibitem{02早川智彦05}
Yuki Kubota,Tomohiko Hayakawa,Osamu Fukayama,and Masatoshi Ishikawa:Sequential estimation of psychophysical parameters based on the paired comparisons,2022 IEEE/SICE International Symposium on System Integration (SII 2022),pp.150-154,2022.

\bibitem{02早川智彦06}
Ke Yushan,Yushi Moko,Yuka Hiruma,Tomohiko Hayakawa,and Masatoshi Ishikawa:Silk printed retroreflective markers for infrastructure maintenance vehicles in tunnels,SPIE Smart Structures + NDE On Demand,2022 (accepted).

\bibitem{03黄守仁02}
Mamoru Oka,Kenichi Murakami,Shouren Huang,Hirofumi Sumi,Masatoshi Ishikawa and Yuji Yamakawa:High-speed Manipulation of Continuous Spreading and Aligning a Suspended Towel-like Object,2022 IEEE/SICE International Symposium on System Integration,2022.

\bibitem{04末石智大06}
Tomohiro Sueishi and Masatoshi Ishikawa:  Ellipses Ring Marker for High-speed Finger Tracking,The 27th ACM Symposium on Virtual Reality Software and Technology (VRST2021) (Osaka),Proceedings,Article No. 31,pp.1-5,2021.


\bibitem{04末石智大07}
Soichiro Matsumura,Tomohiro Sueishi,Shoji Yachida,and Masatoshi Ishikawa:Eye Vibration Detection Using High-speed Optical Tracking and Pupil Center Corneal Reflection,The 43rd Annual International Conference of the IEEE Engineering in Medicine and Biology Society (EMBC2021) (Virtual)/Proceedings,ThDT3.5,2021.

\bibitem{04末石智大08}
Ayumi Matsumoto,Tomohiro Sueishi,and Masatoshi Ishikawa:High-speed Gaze-oriented Projection by Cross-ratio-based Eye Tracking with Dual Infrared Imaging,2022 IEEE Conference on Virtual Reality and 3D User Interfaces Abstracts and Workshops (VRW2022),Proceedings,pp.594-595,2022.

\bibitem{04末石智大09}
Tomohiro Sueishi,Soichiro Matsumura,Shoji Yachida,and Masatoshi Ishikawa: Optical and Control Design of Bright-pupil Microsaccadic Artificial Eye,2022 IEEE/SICE International Symposium on System Integration (SII2022) Online,Proceedings,pp.760-765,2022.

\bibitem{05宮下令央04}
Leo Miyashita,Kentaro Fukamizu,and Masatoshi Ishikawa:Simultaneous Augmentation of Textures and Deformation Based on Dynamic Projection Mapping,SIGGRAPH Asia 2021,Real Time Live!,2021.

\bibitem{05宮下令央05}
Leo Miyashita,Kentaro Fukamizu,and Masatoshi Ishikawa: Simultaneous Augmentation of Textures and Deformation Based on Dynamic Projection Mapping,SIGGRAPH Asia 2021,Emerging Technologies,2021.

\bibitem{05宮下令央06}
Leo Miyashita,Yohta Kimura,Satoshi Tabata,and Masatoshi Ishikawa:High-speed simultaneous measurement of depth and normal for real-time 3D reconstruction,SPIE Optical Engineering + Applications,2021.

\bibitem{06田畑智志02}
Yuping Wang,Senwei Xie,Lihui Wang,Hongjin Xu,Satoshi Tabata,and Masatoshi Ishikawa:ARSlice: Head-Mounted Display Augmented with Dynamic Tracking and Projection,The 10th international conference on Computational Visual Media (CVM 2022),2022(accepted).

\bibitem{09李ソ賢01}
Seohyun Lee,Hyuno Kim,Hideo Higuchi,and Masatoshi Ishikawa:Classification of Metastatic Breast Cancer Cell Using Deep Learning Approach,2021 International conference on artificial intelligence in information and communication (ICAIIC),Proceedings,pp.425-428,2021.

\bibitem{kobayashi2-1}
Keijiro Nakagawa, Daisuk\'e Shimotoku, Junya Kawase and Hill Hiroki Kobayashi, "Sustainable Wildlife DTN: Wearable Animal Resource Optimization through Intergenerational Multi-hop Network Simulation.", Proceedings of 2021 17th International Conference on Wireless and Mobile Computing, Networking and Communications (IEEE WiMob), 2021.
\bibitem{kobayashi2-2}
Daisuk\'e Shimotoku, Junya Kawase, Herv\'e Glotin and Hill Hiroki Kobayashi, "Comparison Between Manual and Automated Annotations of Eco-Acoustic Recordings Collected in Fukushima Restricted Zone.", Proceedings of 2021 International Conference on Human-Computer Interaction (HCII 2021), 2021.  
\bibitem{ykuga33191433}
Yukito Ueno, Ryo Nakamura, Yohei Kuga, Hiroshi Esaki, P2PNIC: High-Speed Packet Forwarding by Direct Communication between NICs, IEEE INFOCOM 2021 - IEEE Conference on Computer Communications Workshops (INFOCOM WKSHPS), pp1-6, May, 2021.

\bibitem{ykuga32183289}
Hajime Tazaki, Akira Moroo, Yohei Kuga, Ryo Nakamura, How to design a library OS for practical containers?, Proceedings of the 17th ACM SIGPLAN/SIGOPS International Conference on Virtual Execution Environments, 16 Apr, 2021.

\bibitem{JIANG2-1}
Zhaonan Wang, Renhe Jiang, Hao Xue, Flora Salim, Xuan Song, Ryosuke Shibasaki, "Event-Aware Multimodal Mobility Nowcasting", Proceedings of Thirty-Sixth AAAI Conference on Artificial Intelligence (AAAI), 2022.
\bibitem{JIANG2-2}
Zhaonan Wang, Renhe Jiang, Zekun Cai, Zipei Fan, Xin Liu, Kyoung-Sook Kim, Xuan Song, Ryosuke Shibasaki, "Spatio-Temporal-Categorical Graph Neural Networks for Fine-Grained Multi-Incident Co-Prediction", Proceedings of 30th ACM International Conference on Information and Knowledge Management (CIKM), 2021.
\bibitem{JIANG2-3}
Renhe Jiang, Du Yin, Zhaonan Wang, Yizhuo Wang, Jinliang Deng, Hangchen Liu, Zekun Cai, Jinliang Deng, Xuan Song, Ryosuke Shibasaki, "DL-Traff: Survey and Benchmark of Deep Learning Models for Urban Traffic Prediction", Proceedings of 30th ACM International Conference on Information and Knowledge Management (CIKM), 2021.
\bibitem{JIANG2-4}
Renhe Jiang, Zhaonan Wang, Zekun Cai, Chuang Yang, Zipei Fan, Tianqi Xia, Go Matsubara, Hiroto Mizuseki, Xuan Song, Ryosuke Shibasaki, "Countrywide OD Matrix Prediction for COVID-19", Proceedings of the European Conference on Machine Learning and Principles and Practice of Knowledge Discovery in Databases (ECML PKDD), 2021. 
\bibitem{JIANG2-5}
Jinliang Deng, Xiusi Chen, Renhe Jiang, Xuan Song, Ivor W. Tsang, "ST-Norm Spatial and Temporal Normalization for Multi-variate Time Series Forecasting", Proceedings of 27th ACM SIGKDD Conference on Knowledge Discovery and Data Mining (KDD), 2021.
\bibitem{JIANG2-6}
Zhaonan Wang, Tianqi Xia, Renhe Jiang, Xin Liu, Kyoung-Sook Kim, Xuan Song, Ryosuke Shibasaki, "Forecasting Ambulance Demand with Profiled Human Mobility via Heterogeneous Multi-graph Convolution Network”, Proceedings of the 37th IEEE International Conference on Data Engineering (ICDE), 2021. 
\bibitem{HNakamura2}
Satoshi Okada, Daisuke Miyamoto, Yuji Sekiya, Hiroshi Nakamura, ``New
LDoS Attack in Zigbee Network and its Possible Countermeasures'', The
5th IEEE International Workshop on Big Data and IoT Security in Smart
Computing, 6pages, Aug.  2021

\bibitem{HNakamura3}
Shaswot Shresthamali, Masaaki Kondo, Hiroshi Nakamura, ``Multi-objective
Reinforcement Learning for Energy Harvesting Wireless Sensor Nodes'',
14th IEEE International Symposium on Embedded Multicore Many-core
Systems-on-Chip (MCSoC 2021), 8pages, Dec. 2021

\bibitem{HNakamura4}
Satoshi Okada, Daisuke Miyamoto, Yuji Sekiya, Hideki Takase, Hiroshi
Nakamura, ``LDoS Attacker Detection Algorithms in Zigbee Network'',
Proceedings of the 14th IEEE International Conference on Internet of
Things, Dec. 2021

\bibitem{HNakamura5}
Siyi Hu, Masaaki Kondo, Yuan He, Ryuichi Sakamoto, Hao Zhang, Jun Zhou
and Hiroshi Nakamura, ``GraphDEAR: An Accelerator Architecture for
Exploiting Cache Locality in Graph Analytics Applications'', Proc. of 30
Euromicro International Conference on Parallel, Distributed, and
Network-based Processing, Mar. 2022


\bibitem{suzumura-sc2021}
Venkatesan T. Chakaravarthy, Shivmaran S. Pandian, Saurabh Raje, Yogish Sabharwal, Toyotaro Suzumura, Shashanka Ubaru, 
"Efficient scaling of dynamic graph neural networks". SC2021(The International Conference for High Performance Computing, Networking, Storage, and Analysis)

\bibitem{suzumura-smds21}
Shilei Zhang, Toyotaro Suzumura, Li Zhang, "DynGraphTrans: Dynamic Graph Embedding via Modified Universal Transformer Networks for Financial Transaction Data", IEEE SMDS 2021 (International Conference on Smart Data Services) 
\end{査読付}

%\begin{公開}{1}
%
%
%\end{公開}
%
%\begin{特許}{1}
%
%
%\end{特許}

\begin{発表}{1}

\bibitem{01石川正俊12}
谷内田尚司,並木重哲,小川拓也,細井利憲,石川正俊:高速カメラ物体認識技術を用いた錠剤外観検査装置,製剤機械技術学会誌,Vol.30,No.4,pp.35-40,2021.

\bibitem{02早川智彦07}
栃岡 陽麻里,早川 智彦,石川 正俊:身体感覚と視覚情報にずれが生じる低遅延没入環境におけるターゲットの加速度がユーザへ与える影響,第26回日本バーチャルリアリティ学会大会 (VRSJ2021),論文集,3B2-3,2021.

\bibitem{03黄守仁03}
村上 健一,黄 守仁,石川 正俊,山川 雄司:高速ビジュアルフィードバックを用いた高速3次元位置補償システムの開発,第22回計測自動制御学会システムインテグレーション部門講演会 (SI2021),講演会論文集,pp.1403-1405,2021.

\bibitem{03黄守仁04}
岡衛,村上健一,黄守仁,角博文,石川正俊,山川雄司:面状柔軟物の展開に向けたコーナーの状態認識と把持動作計画,第39回日本ロボット学会学術講演会(RSJ2021),講演会論文集,3F1-01,2021.

\bibitem{03黄守仁05}
上野永遠,黄守仁,石川正俊:上腕の一自由度回転運動に向けた高周波電気刺激フィードバック制御システムの構築,第39回日本ロボット学会学術講演会(RSJ2021),講演会論文集,RSJ2021AC2J1-01,2021

\bibitem{03黄守仁06}
長谷川雄大,黄守仁,山川雄司,石川正俊:閉リンク機構を用いた動的補償モジュールの開発,第39回日本ロボット学会学術講演会(RSJ2021),講演会論文集,RSJ2021AC2D2-05,2021.

\bibitem{04末石智大10}
末石智大,石川正俊:手指高速トラッキングに向けた楕円群指輪マーカーの開発,第22回計測自動制御学会システムインテグレーション部門講演会(SI2021),講演会論文集,pp.1382-1387,2021.

\bibitem{04末石智大11}
末石智大,松村蒼一郎,谷内田尚司,石川正俊:マイクロサッカード高精度計測に向けた動的な明瞳孔眼球模型の開発,第22回計測自動制御学会システムインテグレーション部門講演会(SI2021),講演会論文集,pp.2011-2016,2021.

\bibitem{04末石智大12}
松村蒼一郎,末石智大,井上満晶,谷内田尚司,石川正俊:光学系制御撮影下の角膜反射法によるマイクロサッカード検出高精度化の検討,第22回計測自動制御学会システムインテグレーション部門講演会(SI2021),講演会論文集,pp.2025-2028,2021.

\bibitem{04末石智大13}
三河祐梨,末石智大,石川正俊:球体姿勢に対応した回転相殺テクスチャの高速投影の残像効果による一軸回転可視化法の提案,第26回日本バーチャルリアリティ学会大会(VRSJ2021),論文集,2D2-5,2021.

\bibitem{04末石智大14}
三河祐梨,末石智大,渡辺義浩,石川正俊:VarioLight2円周マーカを用いた球体への広域かつ遮蔽に頑健なダイナミックプロジェクションマッピング,第27回画像センシングシンポジウム(SSII2021),講演論文集 IS1-25,2021.

\bibitem{ykuga36616798}
塙敏博, 中村遼, 空閑洋平, 杉木章義, 田浦健次朗, データ利活用に向けた仮想化プラットフォームmdxの基本性能評価, 研究報告ハイパフォーマンスコンピューティング(HPC), 7, pp1-9, Mar, 2022.

\bibitem{ykuga36616852}
中村遼, 空閑洋平, AES67のソフトウェアによる実装の試行, 研究報告インターネットと運用技術(IOT), 54, pp1-4, Feb, 2022.

\bibitem{HNakamura6}
仮屋 郷佑, 坂本 龍一, 中村 宏, ``動的スケジューリングによるマイクロサー
ビスの実行最適化'', 情報処理学会コンピュータシステム・シンポジウム論文
集, 2021,pages. 1-10 (2021-11-25)


\bibitem{suzumura-axies2021}
鈴村豊太郎, "データ活用社会創成プラットフォームmdxの設計・実装・運用〜多様な学際領域における共創に向けて~", 大学ICT推進協議会2021年度年次大会(AXIES2021), 2021年12月
\end{発表}

%\begin{特記}{1}
%
%
%\end{特記}

\begin{報道}{1}

\bibitem{01石川正俊13}
石川正俊:「ロボット」100年で次へ 東大特任教授・石川正俊氏に聞く 人のはるかに先を行く 機械性能極限まで発揮,電波新聞,令和3年7月8日.

\bibitem{01石川正俊14}
石川正俊:「高速反応や自律航行 ロボット研究進む 東大でオンライン公開講座」,電波新聞,令和3年6月17日.

\bibitem{01石川正俊15}
石川グループ研究室:MBS毎日,日曜日の初耳学,「VarioLight2,ダイナミックプロジェクションマッピング」,令和3年9月13日.

\bibitem{01石川正俊16}
石川グループ研究室:テレビ東京,日経ニュースプラス9,「ElaMorph projection,VarioLight2,高速トンネル検査」,令和3年6月8日.

\bibitem{01石川正俊17}
石川グループ研究室:TBS,あさチャン!,「ジャンケンロボット,高速道路トンネル検査」,令和3年6月28日.


\end{報道}


\section{データ科学研究部門 教員研究活動}

\subsection{研究報告(小林 博樹)}
\begin{受賞}{1}
\bibitem{kobayashi1-1}
小林博樹:情報通信技術の導入が困難なインフラ圏外空間を対象とした情報デザインとIoTの研究,ドコモ・モバイル・サイエンス賞 社会科学部門 優秀賞,2021/9.

\bibitem{kobayashi1-2}
Hill Hiroki Kobayashi, Radioactive Live Soundscape, Winner, Universal Design Expert, Institute for Universal Design KG, Germany, 2021/05.
\end{受賞}


\begin{査読付}{1}
\bibitem{kobayashi2-1}
Keijiro Nakagawa, Daisuk\'e Shimotoku, Junya Kawase and Hill Hiroki Kobayashi, "Sustainable Wildlife DTN: Wearable Animal Resource Optimization through Intergenerational Multi-hop Network Simulation.", Proceedings of 2021 17th International Conference on Wireless and Mobile Computing, Networking and Communications (IEEE WiMob), 2021.
\bibitem{kobayashi2-2}
Daisuk\'e Shimotoku, Junya Kawase, Herv\'e Glotin and Hill Hiroki Kobayashi, "Comparison Between Manual and Automated Annotations of Eco-Acoustic Recordings Collected in Fukushima Restricted Zone.", Proceedings of 2021 International Conference on Human-Computer Interaction (HCII 2021), 2021.  
\end{査読付}

%半ページから1ページが文量

\subsection{研究報告(鈴村 豊太郎)}

 本節では2021 年度の鈴村豊太郎の研究活動について報告する. 2021年4月に本学に着任し, グラフ構造に関するニューラルネットワークを用いた表現学習 Graph Neural Network (以下、GNNと呼ぶ)の基礎研究及びその様々な応用研究に取り組んでいる. グラフ構造は, ノードと, ノード同士を接続するエッジから構成されるデータ構造である. インターネット上における社会ネットワーク, 購買行動, サプライチェーン, 金融における決済データ, 交通ネットワーク, 蛋白質相互作用・神経活動・DNAシーケンス配列内の依存性, 物質の分子構造, 人間の骨格ネットワーク, 概念の関係性を表現した知識グラフなど, グラフ構造として表現できる応用先は枚挙に暇がない。
\par
当該研究領域において、時系列・動的に変化する大規模グラフに対するGNNモデルの研究を行った。分散計算環境においてスケールするGNNモデルを提唱し、その成果は高性能計算分野におけるトップカンファレンスSC2021\cite{suzumura-sc2021}に採択された. また、金融領域における不正検知手法として、TransformerアーキテクチャをベースにしたGNN手法を提案し、国際会議 IEEE SMDS 2021\cite{suzumura-smds21}に採択された. また, GNNに関する招待講演\cite{suzumura-canon2021}を行った。

これらの研究に続いて、推薦システムへのGNNモデルに関する研究を開始している。実データ・実問題に基づいた、社会実装を見据えた研究を進めるべく、医療・介護領域における人材推薦としてエス・エム・エス社、自動車における経路推薦としてトヨタ社と共同研究を2023年4月から本格的に開始する。また、国立研究開発法人物質・材料研究機構NIMSが主導する「マテリアル先端リサーチインフラ」プロジェクトの本学拠点の一貫で、材料情報科学 Materials Informaticsへの研究も開始している。
 データ科学・データ利活用のためのクラウド基盤 mdx プロジェクトにおいて, 今年度は 課金付き運用開始に向けたシステム拡張,スポットVM、データ共有機構(Platform-as-a-Service)に向けた設計を進めた. また, mdxに関する講演活動を国内外において行った\cite{suzumura-axies2021,suzumura-nanotec2021, suzumura-nci2021}. mdxの論文においては、国際会議IEEE IC2E2022(10th IEEE International Conference on Cloud Engineering) に2022年3月末に投稿した. 
%\cite{suzumura-mdx2022}においても論文を公開した。




% 
% bibitem を作る

\begin{査読付}{1}
\bibitem{suzumura-sc2021}
Venkatesan T. Chakaravarthy, Shivmaran S. Pandian, Saurabh Raje, Yogish Sabharwal, Toyotaro Suzumura, Shashanka Ubaru, 
"Efficient scaling of dynamic graph neural networks". SC2021(The International Conference for High Performance Computing, Networking, Storage, and Analysis)

\bibitem{suzumura-smds21}
Shilei Zhang, Toyotaro Suzumura, Li Zhang, "DynGraphTrans: Dynamic Graph Embedding via Modified Universal Transformer Networks for Financial Transaction Data", IEEE SMDS 2021 (International Conference on Smart Data Services) 
\end{査読付}

\begin{発表}{1}
\bibitem{suzumura-axies2021}
"データ活用社会創成プラットフォームmdxの設計・実・運用〜多様な学際領域における共創に向けて~", 大学ICT推進協議会2021年度年次大会(AXIES2021), 2021年12月15日
\end{発表}

\begin{招待講演}{1}

%\bibitem{suzumura-mdx2022}
%Toyotaro Suzumura, Akiyoshi Sugiki, Hiroyuki Takizawa, Akira Imakura, Hiroshi Nakamura, Kenjiro Taura, Tomohiro Kudoh, Toshihiro Hanawa, Yuji Sekiya, Hiroki Kobayashi, Shin Matsushima, Yohei Kuga, Ryo Nakamura, Renhe Jiang, Junya Kawase, Masatoshi Hanai, Hiroshi Miyazaki, Tsutomu Ishizaki, Daisuke Shimotoku, Daisuke Miyamoto, Kento Aida, Atsuko Takefusa, Takashi Kurimoto, Koji Sasayama, Naoya Kitagawa, Ikki Fujiwara, Yusuke Tanimura, Takayuki Aoki, Toshio Endo, Satoshi Ohshima, Keiichiro Fukazawa, Susumu Date, Toshihiro Uchibayashi, "mdx: A Cloud Platform for Supporting Data Science and Cross-Disciplinary Research Collaborations", https://arxiv.org/abs/2203.14188

\bibitem{suzumura-nci2021}  “mdx: A platform for the data-driven future”、オーストラリア国立研究所NCI(National Computational Infrastructure)-Fujitsu HPC, Cloud and Data Futures Workshop, 2022年02月08日

\bibitem{suzumura-nanotec2021}  "データ活用社会創成プラットフォームmdxにおけるマテリアルズ・インフォマティクス研究・共創に向けて", 第20回ナノテクノロジー総合シンポジウム, 2022年01月28日

\bibitem{suzumura-canon2021}  "人工知能を支えるグラフニューラルネットワークの最新動向", 2021年度キャノングローバル戦略研究所主催「経済・社会との分野横断的研究会」, 2021年12月23日


\end{招待講演}




\subsection{研究報告(松島 慎)}
\input{Matsushima/ITCannual-list-Matsushima}

\subsection{研究報告(空閑 洋平)}
\begin{発表}{2}
\bibitem{ykuga36616798}
塙敏博, 中村遼, 空閑洋平, 杉木章義, 田浦健次朗, データ利活用に向けた仮想化プラットフォームmdxの基本性能評価, 研究報告ハイパフォーマンスコンピューティング(HPC), 7, pp1-9, Mar, 2022.

\bibitem{ykuga36616852}
中村遼, 空閑洋平, AES67のソフトウェアによる実装の試行, 研究報告インターネットと運用技術(IOT), 54, pp1-4, Feb, 2022.

\end{発表}

\begin{査読付}{2}
\bibitem{ykuga33191433}
Yukito Ueno, Ryo Nakamura, Yohei Kuga, Hiroshi Esaki, P2PNIC: High-Speed Packet Forwarding by Direct Communication between NICs, IEEE INFOCOM 2021 - IEEE Conference on Computer Communications Workshops (INFOCOM WKSHPS), pp1-6, May, 2021.

\bibitem{ykuga32183289}
Hajime Tazaki, Akira Moroo, Yohei Kuga, Ryo Nakamura, How to design a library OS for practical containers?, Proceedings of the 17th ACM SIGPLAN/SIGOPS International Conference on Virtual Execution Environments, 16 Apr, 2021.

\end{査読付}

\begin{雑誌論文}{1}
\bibitem{ykuga36595746}
Yukito Ueno, Ryo Nakamura, Yohei Kuga, Hiroshi Esaki, A NIC-driven Architecture for High-speed IP Packet Forwarding on General-purpose Servers, Journal of Information Processing, 30, pp226-237, 2022.

\end{雑誌論文}

\begin{招待講演}{2}
\bibitem{ykuga36619767}
空閑洋平, mdx: データ活用社会創成プラットフォーム構築の現状と今後, CloudWeek 2021@Hokkaido University, 2 Sep, 2021.

\bibitem{ykuga36619729}
空閑洋平, NetTLP: ハードウェアと協調動作可能なソフトウェアPCIeデバイス開発環境, 情報処理学会 FIT 情報科学技術フォーラム トップコンファレンスセッション, 26 Aug, 2021.

\end{招待講演}


\subsection{研究報告(姜 仁河)}
 本節では2021年度の姜仁河の研究活動について報告する。近年、都市のスマート管理、スマートシティは新しい科学技術分野として各国の学術界、産業界および各国政府から非常に重視されている。モバイルデータ、IoTセンサデータ、衛星画像、交通プローブデータ、災害データなどダイナミックなリアルタイム時空間ビッグデータが入手可能な環境が急速に整いつつあり、健康や医療サービスデータ、購買履歴データ、物流・商流などの経済データも積極的に活用されている。これらのデータを統合した形で人々や企業の活動、交通・物流・商流から都市の拡大・環境変化、社会経済システムの変質・変動までを包含するデジタル社会空間のあらゆる課題を解決する。これを目的にして、引き続き2021年度、私は時空間データインテリジェンスについて研究活動を行ってきた。

時空間データを一定の時間間隔・空間単位で集計すると、汎用的に3Dテンソル「T×N×C」(Tはタイムステップの数、Nは交差点やリンクやメッシュグリッドや不規則な区域の数、Cは情報量の数)で表現できる。時間軸Tから見ると、将来の予測値は最近の観測値と過去に現れた定期的なパターンに依存する。空間軸Nから見ると、ある交差点や区域内の交通量は近隣や遠くにあるものに影響される。このような3Dテンソル「T×N×C」に含まれる超複雑な時空間依存関係をモデリングするために、一連の深層学習モデル(Temporal Convolution Network、Recurrent Neural Network 、Graph Convolution Network、Attention Mechanism)を設計・開発し、群衆密度、タクシー・シェアサイクル需要、救急車需要、新型コロナ感染者数を高精度で予測可能にした。関連成果はデータサイエンス分野のトップカンファレンスKDD2021、ICDE 2021、ECMLPKDD2021、CIKM2021、AAAI2022\cite{JIANG2-1,JIANG2-2,JIANG2-3,JIANG2-4,JIANG2-5,JIANG2-6}及びトップジャーナルACM TIST 2022、ACM TKDD 2021、IEEE TKDE2021、IEEE TVCG2021\cite{JIANG1-1,JIANG1-2,JIANG1-3,JIANG1-4,JIANG1-5}にて発表された。特に、交通流モデリング・予測の深層学習ベンチマークDL-Traff\cite{JIANG2-3}は、国際トップカンファレンスCIKM 2021において最優秀リソース論文賞(Best Resource Paper Runner-Up Award)を受賞した。また、Yahoo Japan Researchとの共同研究によって、オープンデータサイエンスの促進に努めた。個人情報・プライバシー問題に配慮したうえで、メッシュベースの東京都・大阪市の群衆密度・移動データをデータサイエンス分野トップジャーナルであるIEEE TKDE2021\cite{JIANG1-4}において公開した。


\begin{雑誌論文}{1}
\bibitem{JIANG1901}
Zipei Fan, Xuan Song, Renhe Jiang, Quanjun Chen, and Ryosuke Shibasaki:
Decentralized Attention-based Personalized Human Mobility Prediction, Proceedings of the ACM on Interactive Mobile Wearable and Ubiquitous Technologies, Vol.3, No.4, pp1-26, December 2019.
\bibitem{JIANG2001}
Zipei Fan, Xuan Song, Quanjun Chen, Renhe Jiang, Ryosuke Shibasaki, and Kota Tsubouchi:  
Trajectory fingerprint: one-shot human trajectory identification using Siamese network, CCF Transactions on Pervasive Computing and Interaction, 2(2), 113-125, 2020.
\bibitem{JIANG2002}
Renhe Jiang, Quanjun Chen, Zekun Cai, Zipei Fan, Xuan Song, Kota Tsubouchi, and Ryosuke Shibasaki: 
Will You Go Where You Search? A Deep Learning Framework for Estimating User Search-and-Go Behavior, Neurocomputing, 2020.
\bibitem{JIANG2003}
Renhe Jiang, Xuan Song, Zipei Fan, Tianqi Xia, Zhaonan Wang, Quanjun Chen, Zekun Cai, and Ryosuke Shibasaki: 
Transfer Urban Human Mobility via POI Embedding over Multiple Cities, ACM/IMS Trans. Data Sci. 2, 1, Article 4, 26 pages, January 2021.
\end{雑誌論文}

\begin{査読付}{1}
\bibitem{JIANG1902}
Renhe Jiang, Xuan Song, Dou Huang, Xiaoya Song, Tianqi Xia, Zekun Cai, Zhaonan Wang, Kyoung-Sook Kim, and Ryosuke Shibasaki:
Deepurbanevent: A system for predicting citywide crowd dynamics at big events, Proceedings of The 25th ACM SIGKDD International Conference on Knowledge Discovery \& Data Mining (KDD'19), pp2114-2122, July 2019.
\bibitem{JIANG1903}
Zipei Fan, Quanjun Chen, Renhe Jiang, Ryosuke Shibasaki, Xuan Song, and Kota Tsubouchi:
Deep Multiple Instance Learning for Human Trajectory Identification, Proceedings of the 27th ACM SIGSPATIAL International Conference on Advances in Geographic Information Systems (SIGSPATIAL'19), pp512-515, November 2019.
\bibitem{JIANG1904}
Xiaodan Shi, Xiaowei Shao, Zipei Fan, Renhe Jiang, Haoran Zhang, Zhiling Guo, Guangming Wu, Wei Yuan, and Ryosuke Shibasaki:
Multimodal Interaction-Aware Trajectory Prediction in Crowded Space, Proceedings of The Thirty-Fourth AAAI Conference on Artificial Intelligence (AAAI'20), pp11982-11989, February 2020.
\bibitem{JIANG2004}
Satoshi Miyazawa, Xuan Song, Renhe Jiang, Zipei Fan, Ryosuke Shibasaki, and Taisei Sato:
City-Scale Human Mobility Prediction Model by Integrating Gnss Trajectories and Sns Data Using Long Short-Term Memory, ISPRS Annals of the Photogrammetry, Remote Sensing and Spatial Information Sciences, Volume V-4-2020, 2020, pp.87-94, August 2020.
\bibitem{JIANG2005}
Quanjun Chen, Renhe Jiang, Chuang Yang, Zekun Cai, Zipei Fan, Kota Tsubouchi, Xuan Song, Ryosuke Shibasaki: 
DualSIN: Dual Sequential Interaction Network for Human Intentional Mobility Prediction, Proceedings of the 28th International Conference on Advances in Geographic Information Systems (SIGSPATIAL '20), pp.283–292, November 2020.
\bibitem{JIANG2006}
Xiaodan Shi, Xiaowei Shao, Guangming Wu, Haoran Zhang, Zhiling Guo, Renhe Jiang, Ryosuke Shibasaki: 
Social-DPF: Socially acceptable distribution prediction of futures, Proceedings of The Thirty-Fifth AAAI Conference on Artificial Intelligence (AAAI'21), February 2021.
\end{査読付}




\subsection{研究報告(川瀬 純也)}
\input{Kawase/ITCannual-list-Kawase}

\subsection{研究報告(華井 雅俊)}

 本節では、2021年度の華井雅俊の研究活動について報告する。2021年9月の本学着任から、グラフニューラルネットワークとその物性予測問題への応用に関する研究に取り組んでいる。

電池、半導体、触媒、医薬品などの材料開発・材料研究の全般において、膨大にある候補材料のさまざまな物性を比較解析することが不可欠であるが、それら候補全てを実際に作り検証することは現実的でない。そのため分子構造などの比較的簡単に得られる物質情報から目的の物性を予測・計算することが重要である。近年では、分子構造(グラフ)データとグラフニューラルネットワークを利用した物性値予測モデルの研究が盛んになってきている。特に2021年度はStanford Universityが取りまとめるOpen Graph Benchmark (OGB) やCMUとFacebookが主導するOpen Catalyst Project (OCP) などの機械学習系研究コミュニティのコンペティションで物性予測問題が取り上げられた初めての年であった。

一般に、ある物性値が広範囲な材料群に対し既知である場合予測モデルを構築することが可能となるが、しかし一方で、多くの物性値においては既知である材料が少数であり学習データが不足しているため、実用精度の予測モデルを構築することは難しい。同一の物性であってもパラメータや実験条件が共通化されていないと予測モデルの構築は難しいことが知られ、既存の物性予測の研究では、共通の条件で整理された大規模データが主に利用される(例えば、上のコンペティションなど)。小規模に限定されるデータ、例えば計算コストの膨大なシミュレーション値や実験データ、において、機械学習の利用は限定的であり、大きな研究課題の1つとなっている。

我々の研究チームはこのような少規模データに着目し研究を開始した。2021年度下半期は新手法提案への準備としてデータの収集に注力し研究を行った。機械学習分野や材料研究分野で用いられるオープンデータに加え、同学の工学部の研究チームへコンタクトし、スパコンスケールの計算資源を利用し得られた高価なシミュレーション値や実際の実験データに関してヒアリングを行い、データ収集を開始した。
また、本部門で開発のすすめるmdxにおいては材料系研究への利用促進を行っており、本研究の中間報告として第20回ナノテクノロジー総合シンポジウムにて発表し、IEEE IC2E 2022への投稿論文にて材料系研究におけるクラウド基盤の利活用をまとめた。

% 2021年度は主に、分野の調査と




\input{Hanai/ITCannual-list-Hanai}

% Ishikawa group

\end{document}

