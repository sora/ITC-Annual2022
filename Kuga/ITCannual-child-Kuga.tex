\subsection{研究報告(空閑 洋平)}
 現在のデータセンタ環境では、機械学習やニューラルネットワークの学習,推論を高速化する専用アクセラレータが広く使用されるようになった。専用アクセラレータを用いた計算環境は、既存のCPUを中心に構成したソフトウェア環境に比べて、プロセッサやデバイスドライバ、デバイス間通信が専用に設計されており、ユーザによる状態の把握や可視化が困難なブラックボックス化が進んでいる。今後、専用アクセラレータ、メモリ、NIC、ストレージなどのデータセンタデバイスにプロセッサが搭載されることになれば、セキュリティ監視や脆弱性試験、管理手法、データ通信内容の可視化といった、ソフトウェアと同様の課題が顕在化すると考えられる。

今年度は、このようなデータセンタハードウェアの脆弱性試験手法に関する研究を開始した。また、昨年度から引き続きNICを中心としたデータセンタハードウェアの高性能化、高機能化に関する研究を行い\cite{ykuga33191433, ykuga32183289, ykuga36595746}、招待講演を一件実施した\cite{ykuga36619729}。データ科学・データ利活用のためのクラウド基盤mdxプロジェクトでは、構築した仮想マシン環境上のストレージ性能の調査を実施\cite{ykuga36616798}し、北海道大学主催のCloudweekではmdxを紹介した\cite{ykuga36619767}。また、コロナ禍によって遠隔講義、リモートワークが広く社会に普及したことで音響通信技術の重要性が増しており、音響デバイスのIP化の技術検討を開始し、その初期段階の進捗を研究会で報告した\cite{ykuga36616852}。