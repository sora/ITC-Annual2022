%半ページから1ページが文量

\subsection{研究報告(鈴村 豊太郎)}

 本節では2021 年度の鈴村豊太郎の研究活動について報告する。2021年4月に本学に着任し、グラフ構造に関するニューラルネットワークを用いた表現学習 Graph Neural Network (以下、GNNと呼ぶ)の基礎研究及びその様々な応用研究に取り組んでいる。グラフ構造は、ノードと、ノード同士を接続するエッジから構成されるデータ構造である。インターネット上における社会ネットワーク、購買行動、サプライチェーン、金融における決済データ、交通ネットワーク、蛋白質相互作用・神経活動・DNAシーケンス配列内の依存性、物質の分子構造、人間の骨格ネットワーク、概念の関係性を表現した知識グラフなど、グラフ構造として表現できる応用先は枚挙に暇がない。
\par
 当該研究領域において、時系列・動的に変化する大規模グラフに対するGNNモデルの研究を行った。分散計算環境においてスケールするGNNモデルを提唱し、その成果は高性能計算分野におけるトップカンファレンスSC2021\cite{suzumura-sc2021}に採択された。また、金融領域における不正検知手法として、TransformerアーキテクチャをベースにしたGNN手法を提案し、国際会議 IEEE SMDS 2021\cite{suzumura-smds21}に採択された。また、GNNに関する招待講演\cite{suzumura-canon2021}を行った。

 これらの研究に続いて、推薦システムへのGNNモデルに関する研究を開始している。実データ・実問題に基づいた、社会実装を見据えた研究を遂行するため、医療・介護領域における人材推薦としてエス・エム・エス社、自動車における経路推薦としてトヨタ社と共同研究を開始している。また、「マテリアル先端リサーチインフラ」プロジェクトの一貫で、材料情報科学 Materials Informaticsへの研究も開始している。
 データ科学・データ利活用のためのクラウド基盤 mdx プロジェクトにおいては、 課金付き運用開始に向けたシステム拡張,スポットVM、データ共有機構(Platform-as-a-Service)に向けた設計を進めた。また、mdxに関する講演活動を国内外において行った\cite{suzumura-axies2021,suzumura-nanotec2021, suzumura-nci2021}。 mdxの論文においては、国際会議IEEE IC2E2022に2022年3月末に投稿した。
%\cite{suzumura-mdx2022}においても論文を公開した。

