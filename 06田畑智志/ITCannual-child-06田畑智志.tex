\subsection{研究報告(田畑 智志)}

高速な三次元計測および可変焦点投影に関する研究

物体の三次元的な形状・運動の高速取得や、三次元空間に対する情報の高速フィードバックは、現実空間とデジタル空間のインタラクションをはじめ、多くの応用の基盤となる技術である。特に、形状・運動情報の統合による広範囲の形状復元はアーカイブや環境認識において重要である。そこで、これまで開発してきた技術の統合を進め高速三次元スキャンの安定化を進めている。また、三次元空間に情報を提示する高速焦点追従投影システムにおいても応用展開を進めている。

高速三次元形状計測においては、これまでに開発しているセグメントパターンによる小型形状計測ユニットを用いた1,000fpsでの物体の高速三次元スキャン技術を安定化させる技術開発を行っている。そこで、従来取り入れていた運動計測に加えて、計測データを統合したモデル生成とモデルベースの運動補正を組み合わせ、広範囲計測におけるドリフト補正の開発を進めている。特に、制約条件や処理速度のトレードオフに対して異なるフレームレートで動作する処理を組み合わせることで、処理速度と安定した計測の両立を実現している。また、三次元形状計測技術の解像度・精度面の向上をはかるため複数のアプローチに取り組み、高解像度・高精度計測手法の検討を進めている。

高速焦点追従投影システムでは、投影対象の位置姿勢を高速に計測し、その情報をフィードバックして液体レンズと高速プロジェクタを制御するシステムを構築している。特に応用技術として、マーカを付与した簡易的なヘッドマウントディスプレイに対して適用し、プロジェクションによる映像提示を行うことが可能なデバイスの技術開発を行った。

